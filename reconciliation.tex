
\chapter*{Reconciliation}
\label{cha:reconciliation}

\section*{Overview}
\label{sec:overview}

In a world where both cyborgs and genetically modified humans are living in the same place, forcing people to choose to go for one side or the other instead of taking the good of each one of them would certainly create a situation of inequality, which could lead to conflicts and tensions.\\

The two sides of the battle do not defend positions that are mutually exclusive, or symmetrical extremes of the subject of augmentation by themselves. It’s therefore important to work together to create a new scenario where people can freely choose, according to the situation, what kind of technology is the most suitable. From this conjunction point, our aim is thus that of exploring high level regulations and boundaries to maximize the added value of human augmentations, with respect to both the enhancement methods.\\

Imagine a world where everyone can permanently cure a genetic disease with a simple DNA modification, preventing this from emerging in future generations or imagine to fix a disability like legs dysfunction by replacing them in a few hours with a set of robotic legs that are identical to the normal one. Imagine a world where the concept of work is revolutionized, where humans can interact with machines and computer systems in a much more efficient way as if they’re talking to another human or a stage where humans can access and discover all the environments that were previously considered hazardous and inaccessible. Imagine a world where people can finally better understand each other, strengthen their relationships and rediscover themselves and who’s around them.\\

All of this can be achieved if society works side by side in strengthening and combining genetic modification with cyborgization. 

\section*{Demographics}
\label{sec:demographics}

As far as the shape and figures of the population are concerned, no significant opposing arguments would arise. It is reasonable to foresee that through enhancement technologies, people will be able to live much longer: the combination of cyber implants and biological and genetic engineering will render healthcare far more effective in general, slowing overall aging, eradicating several diseases, drastically quickening the response time to health issues and shortening medical recovery times.\\

While the age of the population will rise, so will be their average health and productivity over time, allowing for a general translation to a retribution and retirement system on a more extended period. Such systems being already present, it would certainly avoid disruption and system re-design needs, at least in an initial phase – opening for further developments and integration with automation and wealth distribution possibilities.\\

Concerning the issue of managing a possible overpopulation, as the Genetic Modifications team proposes, optimizing the digestion process could grant the sustainability of global nutrition assets. Besides food, many other societal issues come with overpopulation – managing the job market, the living conditions and the psycho-social implications of highly populated areas are among those. Well devised birth control policies, redistribution and work opportunities policies, urban plans, as well as the collective smartness “engine” that cyber communication may instantiate, could tackle the issue in a way.\\

Indeed, what discussed here assumes a uniform adoption of enhancement technologies. This would be the case if (at least) healthcare-related augmentations are universally available: again, a matter of policy definition and actuation that we have cognition of, so that no evident class A and class B human division is in place.

\section*{Market implications}
\label{sec:market}

Enhancement technology was first born for medical applications [source] and later adapted to further uses. It seems fair to assume that the augmentation market would thus be divided in two main categories: healthcare sector (enhancement to normalize or prevent a deficit) and all the remaining (cognitive and physical enhancements for healthy individuals) [source]. Within the battle constraints of hard-coded mutual exclusion among the two augmentation types, in order to gain appreciation and competitiveness, each side would have had to provide capabilities to reach for both markets. Pragmatically speaking, there is no reason to assume so.\\

Cyber micro-cells could aid in performing biological modification routines, whereas specialized bio-engineering may make the neural communication between body and cyber components more efficient. While cyber parts wouldn’t get illnesses, very much could the rest of the non-cyborg body, so genetic modification-based medications would be of great need in any case. Under a different perspective, having a lost limb or a failing organ regrown would bear disadvantages over having it substituted right away with a synthetic, stronger one.\\

On the second market domain, applications in micro-mobility better suits the cyborg case; overall bodily strength, environmental resilience and adaptation possibilities would scale smoothly with cyber augmentations, but bones and skin reinforcing would provide for a less invasive and general purpose first level of integrity enhancement. Cognitive empowerment would also work on different levels: genetic drugs would provide a less invasive approach to improve focus, wit and perception, while electrodes-based digital brain interfaces would open huge human-human and human-machine communication possibilities, as well as cognitive and computational capabilities to apply in working contexts.\\

With the discussed market subdivision, defining a class of medical-use augmentations would be the first step towards an integration of those in national and international healthcare services; while other context-specific regulations can be devised in the other market.


\section*{Overall Societal Concerns}
\label{sec:rec_concerns}

There is no denial that augmentation technologies will tend to generate a productivity and life quality divide among those who possess them, and those who do not. This is true especially considering the value that an augmented individual has for in the job market over the non augmented people. Once again both sides in the battle suffer from this condition.\\

A way to address such issue is by considering one among three regulatory approaches for augmentation in the work sector: top-down, where it is the Government or a company that provides its employees with augmentations; bottom up, where it is up to each individual; or a mixed approach of the two [cite augmentations and work study]. Both teams reckon that a top-down approach would be sensible where augmentations are not permanent: with the assumption that genetic modification technologies are 100\% reliable, temporary brain and physical enhancements would very well serve the case where an organization offers his workers such option, in order to gain advantage over competitors, and not generating divide among the human workforce.\\

Several are, though, the possibilities of enhanced augmented reality and connectivity that the cyber augmentations offer. Deeper communication levels and human-machine interaction will possibly be of huge value in sectors like engineering and management. Because of ethical issues related to body integrity, a top-down approach would be harder to implement (unless such technologies are detached from the body), but a bottom-up would generate too much divide. Autonomous workers aside, an option for the mixed approach would be a policy for coupling the positions with augmented and non augmented individuals. In order to further empower workers’ rights in a so delicate position, the just mentioned policy can be accompanied by a reform of the trade union, so that the introduction of augmentation shall also address a better protection of workers from any form of exploitation. Addressing the general issue of access divide, we refer to the general considerations each side brought in its view.\\

Further concerns include the constitution of a black and illicit augmentation market. The subject is still shared between the sides: special forces are needed to tackle possible misuse of the technologies. As it happens nowadays, there is no a magic formula to fight against black market; however, strong policies and a specialized security corp will be key to combat it and lead it to its lowest levels. \\

One last point of debate is how to maintain a factual identity metric when body parts are substituted and genes altered. Trace of the mutated genetic information related to an identifying label of an individual (its name) may be kept from the birth (maybe through the use of a blockchain) and stored in a national database. Such data object information can then be physically integrated into the cyber components that an individual may acquire.


\section*{Human relationships}
\label{sec:relationships}

One of the most critical aspects that human enhancement - and all that gravitates around it - touches is certainly the sphere of human relationships. \\

In this field it is important more than ever that cyborg and genetic modification technologies combine their features in the best possible way. This aspect needs to be stressed not only to avoid the creation of unhealthy relationships or inequality, but also to sharply improve the current situation of human relationships.

Nowadays, in this strongly work-money-based society, human beings struggle very often in understanding each other. This happens not only because of the ordinary issues that occur, for instance, within family or couples, but because mental illness - caused by work-related stress and anxiety -  is increasingly becoming a big and central issue. In our future context, improving individual’s quality of life will be a primary aim of augmentations. The endeavour through which such technologies are conveyed shall push towards empathy and fairness. Cognitive enhancement will especially address mental illnesses, and forms of stress that may derive from work and similar environments. This would be achieved in the practical aspect of enhancing brain efficiency and cognitive effort management, and, consequently, with the promise of granting more free time to enjoy and pursue whatever life interest an individual may have.\\

The above, together with inclusion policies, will tackle the issue of psychological subjection and envy that may possibly arise in non augmented individuals, with the hope of smoothing out communication and relationship gaps between anyone, in such shaped society.\\

Cosmetics as well will play a role in relationships: addressing the issue of psychological ease with respect to visual perception and identification of peers (ethnic identity in a future multicultural context) augmentations would have to be aesthetically accepted. The teams agree that physically enhancing technologies (those overall targeting strength and mobility) best apply to the cyborg case, as the provided augmentations could be customized in looks, scalable and easily substituted. On the other hand, “beauty” augmentations – the possibility to change one’s facial and body traits to meet their desires – are best targeted by highly engineered genetic modification means.
