
\chapter*{Scenario} % senza numerazione
\label{cha:scenario}

Imagine: a world where our “body” is a relative concept, where its spatial and functional limits of interaction are overcome, and its possibilities stretched on the axes of electromagnetism and genetic engineering. Each human can choose to be so enhanced: either in the direction of cyber, or genetic modifications.\\ 

In this not so far future, these technologies will be 100\% reliable: installing them bears no chance for errors or mistakes. Its accessibility will be comparable to the one of today’s smartphones: citizens of developed countries will find them easily accessible and affordable, while they’ll be harder to acquire in developing countries. Following the smartphone’s example, the price and access to the modification will be proportional to its complexity and the provided features.\\

Introducing such a change in society will inevitably create key differences from the world as it is today. Although it is too early to determine how this society will end up looking like, it is possible to foresee the sectors that will experience the biggest changes. The concept of work will change, as well as the health system and the human relationships. In a similar context, enhancement technologies also have to address issues such as over and under population, ethical boundaries of augmentations, and maintenance of civil order.\\

An important caveat that characterizes this society has to be taken into consideration: once you decide to get one type of modification - Cyborg or GM - it is not possible getting modifications of the other type. This means that if a human chooses to become a cyborg, he won’t be allowed to switch to genetic modification and vice versa. Therefore, the advantages and limitations of each technology type are a big concern in society, and they have to be clearly explained before the transition process.\\

Within the scope of the battle preparation, a set of rules were agreed upon via a bipartisan process, in order to balance and shape the positions and the domains of the two opposed sides:\\
\begin{itemize}
    \item \textbf{Transferability}. The modifications are a personal choice, therefore they’re not transferable to offspring. Otherwise, the choice of a parent would affect the future of the child, depriving him from his right of choice. This also affects the economic aspect, since making the modifications transferable would decrease the number of customers. 
    \item \textbf{Species Boundaries.} Modifications of animals, food or any other living being is not covered. The teams have to focus exclusively on human enhancement, since going beyond might incur in very complex scenarios and lose the original topic.
    \item \textbf{Brain Modifications.} It is not possible to replace or transfer brains, which eliminates any possibility of transferring consciousness. The reason behind this choice is that allowing this kind of modification could escalate to an unpredictable point and it would threaten the peaceful coexistence of human beings.
    \item \textbf{Military Area.} The aim of the modifications consists in improving human life. There is no contemplation regarding the implementation of such technologies for military applications.
    \item \textbf{Resources and Pollution.} In a world where Cyborgs and GM humans have become common, augmentation technology has reached a level of advancement such that resource consumption and pollution do not represent an issue.
    \item \textbf{Side Domains.}  
        \begin{itemize}
            \item Global coverage: travelling to and/or colonizing other planets is not contemplated. The main goal of this technology is human enhancement, this is done by better adapting to life on Earth not on other planets. Modifications can facilitate space exploration but this cannot be amplified. 
            \item Cloning: as the main objective of both teams is human augmentation, cloning will not be considered by any of them.
        \end{itemize}
\end{itemize}







