
\chapter*{Conclusion}
\label{cha:conclusion}

As the State of the digital and medical Arts progresses on, and the state of the globalized society moves towards an increasingly deeper self-consciousness [there is a source saying this], we may foresee the topic of human augmentations gaining a progressively larger focus. Once again, our sides in the battle - genetic modification and cyborg augmentations - do not constitute mutually exclusive: our reconciliation expresses an effort to highlight some of the most relevant concerns that devising an augmented society should consider.\\

Points of discussion have been provided on the innovative potential that both augmentation technologies bear. Emerging in an incremental fashion from the State of Art basis, enhancing the cognitive and physical capabilities, providing advanced computational and visualization power to the individuals, they offer disruptive possibilities with respect to the actual automation and digitalization trends.  But again, the disruption related to the mere “factual” introduction of such technologies is not set to be so drastic, given how acquainted the society got with innovation: an initial post-digital transformation that human augmentations entail could be smooth. What instead smooth wouldn’t be, as has been noted multiple times throughout the report, is the implementation of a policy structure that should come with enhancements.\\

One point of debate dominates the concerns on augmentations: the divide, the wealth and opportunities gap they may generate [cite the reports on augmentation and work, economic inequality, etc.]. With this knowledge, we reckon that the focus on policies has paramount importance in the matter, possibly more than the augmentation possibilities themselves - and it is, in fact, rather difficult to fully explore. A business-as-usual approach may bear high risks of privileging the richer in enjoying augmentations, bringing about a massive societal division, based on wealth and power, between super-humans and sub-humanized individuals [cite economic inequality and HET]. Fundamental is thus that governments and international entities provide a deep regulatory frame, managing augmentation provisions in working sectors, in healthcare, and augmentation scope of functionalities. Corporate Social Responsibility would be mandatory for any venture participating in the enhancement market, and such products will have to be user and society-oriented. As has been the case for the internet, augmentations would represent a new stage for humanity, and we better get it right.
