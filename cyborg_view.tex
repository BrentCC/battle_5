\chapter*{Cyborg View}
\label{cha:cyborg}

\section*{Mission}
\label{sec:mission}

Technological devices, automated machines, and “intelligent” software are becoming parts of our lives as never before. Apart from having direct experience of that, it is confirmed by several studies and forecasts that digitalization is a trend that we will have ongoing for decades to come [cite sources]. Assuming that such momentum is to persist, in the prospect of about a hundred years from now, the case will be strong for the use of machines and artificial intelligence against any human workforce in the very most of the working sectors [again, a source]. Policymakers – and technology providers themselves - would need to be active in taking the reins of the process so as to avoid concentrations of digital power, possibly unfair redistribution of wealth from automated processes, the overall reliability of digital services, and grant societal sustainability. Following considerations of such kind, what will the role, the ratio for an ordinary human being to partake in society be, is not trivial to predict.\\

An option to render this paradigm-shift easier to manage according to some opinions [cite sources on cyborg trends and possibilities like Elon Musk’s], both under an ethical and an organizational perspective, is to implement new interfaces between humans and technology – by moving towards a deeper communication and cooperation between the two, through the means of cyber augmentations.\\

Two levels of analysis promptly arise to comment on such a possibility. On a lower, “historical” level, bodily and mental interfaces to directly partake in data processing, computation, communications, monitoring, and such, would ease a smooth transition to a world where the potentials of information technology can be enjoyed to the most, as it wouldn’t require re-designing the social structure, at least in first place. On a higher, philosophical and ethical level, cyber augmentations would allow overcoming bodily and mental limits in a more aesthetic sense, towards new realms of amusement, health, communication and collective smartness, all of it maintaining a human resemblance – or giving the freedom to customize it – thus granting the persistence of an identity, visual, in a first instance.\\

This last point, as all of the above, do not so smoothly apply to the case of augmentation through genetic modifications. We thus aim at proposing and promoting the use of cyber augmentation, mindful of an overall societary frame to host it in the best possible way.

\section*{Possibilities}
\label{sec:possibilities}
 
Technology can already be very close to our body, allowing us to access the world of information, communication and computation, just by lifting our hand from our pocket, looking at our left wrist, or clicking a switch on our glasses. Therefore, when supporting a hypothesised concrete societal curiosity towards cyber augmentations, attention must be put into the respective concrete added value that they would provide, and the justification of their use. In our reasoning, an interesting potential for cyber augmentations to enhance possibilities was individuated in the following fields.\\

Healthcare, above all, would be positively disrupted: illnesses do not arise in inorganic matter; synthetic limbs, tissues and organs, neurally connected to the brain, finely engineered with the most advanced materials and methods [reference to like cyber cells and materials], would open for endless possibilities in terms of bodily resistance, systems functionality, resilience, integrity, and environmental adaptation. Overall health would be drastically increased...\\

...As well as mental health and brain power: as the use of non-invasive brain interfaced electrodes is perfectioned [reference to neuralink], areas of the brain can be stimulated to improve perception, focus, receptivity, as to rise the mental capability of the individuals. Not only this, through electrodes and brain-waves scans, mental computation support and visualization interfaces could be implemented, also providing modules for further augmented reality (AR) and communication applications.\\

Immediate sharing of complex mental schemes, AR visualization of contextual information, and mental modelling of virtual reality… these possibilities would offer great empowerment to creativity based roles and jobs like police detectives, engineers and architects, mathematicians, philosophers, artists. Such enhancement would improve mutual human communication and understanding, for the benefit of every user’s human relationship.\\

Lastly, short-distance mobility could be positively changed, wise of augmentations affecting limbs functionalities and strength, for example. Also, integrating generic entertainment and utility components into the cyber parts would be possible.


\section*{Society}
\label{sec:society}

By a surface analysis of a plausible social picture where cyborg augmentations, as we devise them, are widespread, no particular disruptive elements are to be found. The looks of cyber individuals would very closely resemble the human ones, preventing any generation of visual discrimination. Rather, an individual may affordably customize its looks in the way that better fits his personality and desire, with very positive effects on self-confidence [cite studies on looks and self confidence?]. As said, social interactions also would benefit from augmentations, in the way in which brain interfaces shall provide an additional communication level, and improve sensitivity towards a social context - possibly analysing significant psychological elements. Those interfaces in fact, and an encouragement for a higher mindfulness, would represent a further instrument not to arouse incomprehensions.\\

Productivity, quality of life and digital adaptation are the main objectives. Remaining within the frame of the already existing western societal structure [Marx], further digital transformation would come but in the process of integrating augmentations. It would thus be easily manageable in a first place, while raising the living standards, possibly paving the way for further societal development, towards an ideological shift of paradigm for a less work-intensive, sustainable and mindful society.\\

Of course though, a global access divide between those who can and those who cannot afford or access to the augmentation technologies would tend to emerge. Making some individuals more apt to enjoy the new possibilities and their positive effects. Addressing this concern, in our vision, we wish and foresee international entities to regulate the market of augmentations, promoting open innovation, open source projects, research and initiatives to extend the enhancement services, especially the healthcare sector, as widely as possible [As in the case of the internet: cite examples of organizations!].

\section*{Overall Concerns}
\label{sec:concerns}

While many are the possibilities for a globally positive impact of cyber augmentations, some downsides and concerns have to be taken into account.\\

A very first note is that the installation of such components would have to be as little invasive as possible, so that a proper momentum for a similar market would be in place. And in case of components faults, the user must be very promptly provided with proper substitution in order to grant its bodily integrity. With respect to the offered functionalities, there have to be regulations to limit their scope of action, not to generate augmentations that would give an individual powers or offensive potential too complex to manage. Along the same consideration, bodily modifications would have to abide by ethical standards. A proper method to deal with rogue augmentation producers and augmented individuals would have to be set up.\\

Paramount attention has to be put in making sure that no augmentation can be exploited by the manufacturers or malicious parties. Especially for what concerns the brain interfaces and enhancements, hence the promotion of open source technology, research and transparency of implementation.\\

Any component has to work exclusively offline, via inputs that are not but sensory or cerebral. The only exception is with some augmented reality application and brain communication: they must be implemented in a way for which only visual information may be hacked, and a reliable turn-off switch must be present in case of extreme need.

\section*{Final Remark}
\label{sec:remark}

Augmentations of the cyborg type bear a set of advantages over the genetic modification ones. Humanity is preserved, in the looks and in the genetic identity. Besides, any possible genetic modification can be replicated via electromagnetism and synthetic materials, in a way which is faster, more flexible, and more scalable. And even more can be done. Algorithmic and AI power provided to brain interfaces and augmented reality can boost brain efficiency, enhance communication, and “life UX”.\\

Integrating the machine with the man, no human-computer dichotomy is to persist: societary technological disruptions are averted - and Terminator scenarios avoided. Like the advent of the internet, widespread cyber components will bring the human being to reach a higher evolutionary stage, the first pseudo-evolution after the sapiens-sapiens. The humanity will be stronger, more powerful, and socially intelligent [maybe cite a source].\\

The technology may be achieved, the rest is on the policies.

