
%
% Per la generazione corretta del 
% pdflatex nome_file.tex
% bibtex nome_file.aux
% pdflatex nome_file.tex
% pdflatex nome_file.tex


% formato FRONTE RETRO
\documentclass[epsfig,a4paper,11pt,titlepage,oneside,openany]{book}
\usepackage{epsfig}
\usepackage{plain}
\usepackage{setspace}
\usepackage[paperheight=29.7cm,paperwidth=21cm,outer=1.5cm,inner=2.5cm,top=2cm,bottom=2cm]{geometry} % per definizione layout
\usepackage{titlesec} % per formato custom dei titoli dei capitoli
\usepackage{multicol}

%%%%%%%%%%%%%%
% supporto lettere accentate
%
%\usepackage[latin1]{inputenc} % per Windows;
\usepackage[utf8x]{inputenc} % per Linux (richiede il pacchetto unicode);
%\usepackage[applemac]{inputenc} % per Mac.

\singlespacing

\usepackage[english]{babel}

\begin{document}

  % nessuna numerazione
  \pagenumbering{gobble} 
  \pagestyle{plain}

\thispagestyle{empty}

\begin{center}
  \LARGE{Department of Information Engineering and Computer Science (DISI)\\}

  \vspace{1 cm} 
  
  \LARGE{University of Trento, Italy\\}


  \begin{figure}[h]
    \centering
    \includegraphics[height=10cm, width=10cm]{Logo.png}
    \label{fig:dis}
  \end{figure}

  \Large{INNOVATION AND ENTREPRENEURSHIP BASICS - AY 2019/2020\\} 
  \Large{Battle Report for Battle 5}

  \noindent\rule{\textwidth}{1pt}

  \Huge\textsc{\textbf{Cyborg \& Genetical Modification Enhancement\\}}
  \noindent\rule{\textwidth}{1pt}
  
  \begin{multicols}{2}
    \Large{Cyborg\\

    \vspace{1 cm}
    Amer Delibasic[]\\
    Marco Zu[]\\
    Arya Hassanli[]\\
    Luca Morgese[]\\
    Alberto Bellumat[]\\}
  
    \columnbreak
 
    \Large{Genetical Modified\\

    \vspace{1 cm}
    Mouslim Fatnassi[]\\
    Andrea Lorenzo Polo[]\\
    Francesco []\\
    Pop Vasile Adrian Bogdan[]\\
    Mattia Simone[]\\
    Alex Sartori[]\\}
  \end{multicols}
  
  
\end{center}



  \clearpage
 
  
  \pagestyle{plain} % nessuna intestazione e pie pagina con numero al centro

  
  % inizio numerazione pagine in numeri arabi
  \mainmatter      
    \begingroup
 
      \renewcommand{\cleardoublepage}{} 
      \renewcommand{\clearpage}{} 

      \titleformat{\chapter}
        {\normalfont\Huge\bfseries}{\thechapter}{1em}{}
        
      \titlespacing*{\chapter}{0pt}{0.59in}{0.02in}
      \titlespacing*{\section}{0pt}{0.20in}{0.02in}
      \titlespacing*{\subsection}{0pt}{0.10in}{0.02in}
      

      % List sections
      \input{introduction}
      
\chapter*{Scenario} % senza numerazione
\label{cha:scenario}

Imagine: a world where our “body” is a relative concept, where its spatial and functional limits of interaction are overcome, and its possibilities stretched on the axes of electromagnetism and genetic engineering. Each human can choose to be so enhanced: either in the direction of cyber, or genetic modifications.\\ 

In this not so far future, these technologies will be 100\% reliable: installing them bears no chance for errors or mistakes. Its accessibility will be comparable to the one of today’s smartphones: citizens of developed countries will find them easily accessible and affordable, while they’ll be harder to acquire in developing countries. Following the smartphone’s example, the price and access to the modification will be proportional to its complexity and the provided features.\\

Introducing such a change in society will inevitably create key differences from the world as it is today. Although it is too early to determine how this society will end up looking like, it is possible to foresee the sectors that will experience the biggest changes. The concept of work will change, as well as the health system and the human relationships. In a similar context, enhancement technologies also have to address issues such as over and under population, ethical boundaries of augmentations, and maintenance of civil order.\\

An important caveat that characterizes this society has to be taken into consideration: once you decide to get one type of modification - Cyborg or GM - it is not possible getting modifications of the other type. This means that if a human chooses to become a cyborg, he won’t be allowed to switch to genetic modification and vice versa. Therefore, the advantages and limitations of each technology type are a big concern in society, and they have to be clearly explained before the transition process.\\

Within the scope of the battle preparation, a set of rules were agreed upon via a bipartisan process, in order to balance and shape the positions and the domains of the two opposed sides:\\
\begin{itemize}
    \item \textbf{Transferability}. The modifications are a personal choice, therefore they’re not transferable to offspring. Otherwise, the choice of a parent would affect the future of the child, depriving him from his right of choice. This also affects the economic aspect, since making the modifications transferable would decrease the number of customers. 
    \item \textbf{Species Boundaries.} Modifications of animals, food or any other living being is not covered. The teams have to focus exclusively on human enhancement, since going beyond might incur in very complex scenarios and lose the original topic.
    \item \textbf{Brain Modifications.} It is not possible to replace or transfer brains, which eliminates any possibility of transferring consciousness. The reason behind this choice is that allowing this kind of modification could escalate to an unpredictable point and it would threaten the peaceful coexistence of human beings.
    \item \textbf{Military Area.} The aim of the modifications consists in improving human life. There is no contemplation regarding the implementation of such technologies for military applications.
    \item \textbf{Resources and Pollution.} In a world where Cyborgs and GM humans have become common, augmentation technology has reached a level of advancement such that resource consumption and pollution do not represent an issue.
    \item \textbf{Side Domains.}  
        \begin{itemize}
            \item Global coverage: travelling to and/or colonizing other planets is not contemplated. The main goal of this technology is human enhancement, this is done by better adapting to life on Earth not on other planets. Modifications can facilitate space exploration but this cannot be amplified. 
            \item Cloning: as the main objective of both teams is human augmentation, cloning will not be considered by any of them.
        \end{itemize}
\end{itemize}








      \chapter*{Cyborg View}
\label{cha:cyborg}

\section*{Mission}
\label{sec:mission}

Technological devices, automated machines, and “intelligent” software are becoming parts of our lives as never before. Apart from having direct experience of that, it is confirmed by several studies and forecasts that digitalization is a trend that we will have ongoing for decades to come [cite sources]. Assuming that such momentum is to persist, in the prospect of about a hundred years from now, the case will be strong for the use of machines and artificial intelligence against any human workforce in the very most of the working sectors [again, a source]. Policymakers – and technology providers themselves - would need to be active in taking the reins of the process so as to avoid concentrations of digital power, possibly unfair redistribution of wealth from automated processes, the overall reliability of digital services, and grant societal sustainability. Following considerations of such kind, what will the role, the ratio for an ordinary human being to partake in society be, is not trivial to predict.\\

An option to render this paradigm-shift easier to manage according to some opinions [cite sources on cyborg trends and possibilities like Elon Musk’s], both under an ethical and an organizational perspective, is to implement new interfaces between humans and technology – by moving towards a deeper communication and cooperation between the two, through the means of cyber augmentations.\\

Two levels of analysis promptly arise to comment on such a possibility. On a lower, “historical” level, bodily and mental interfaces to directly partake in data processing, computation, communications, monitoring, and such, would ease a smooth transition to a world where the potentials of information technology can be enjoyed to the most, as it wouldn’t require re-designing the social structure, at least in first place. On a higher, philosophical and ethical level, cyber augmentations would allow overcoming bodily and mental limits in a more aesthetic sense, towards new realms of amusement, health, communication and collective smartness, all of it maintaining a human resemblance – or giving the freedom to customize it – thus granting the persistence of an identity, visual, in a first instance.\\

This last point, as all of the above, do not so smoothly apply to the case of augmentation through genetic modifications. We thus aim at proposing and promoting the use of cyber augmentation, mindful of an overall societary frame to host it in the best possible way.

\section*{Possibilities}
\label{sec:possibilities}
 
Technology can already be very close to our body, allowing us to access the world of information, communication and computation, just by lifting our hand from our pocket, looking at our left wrist, or clicking a switch on our glasses. Therefore, when supporting a hypothesised concrete societal curiosity towards cyber augmentations, attention must be put into the respective concrete added value that they would provide, and the justification of their use. In our reasoning, an interesting potential for cyber augmentations to enhance possibilities was individuated in the following fields.\\

Healthcare, above all, would be positively disrupted: illnesses do not arise in inorganic matter; synthetic limbs, tissues and organs, neurally connected to the brain, finely engineered with the most advanced materials and methods [reference to like cyber cells and materials], would open for endless possibilities in terms of bodily resistance, systems functionality, resilience, integrity, and environmental adaptation. Overall health would be drastically increased...\\

...As well as mental health and brain power: as the use of non-invasive brain interfaced electrodes is perfectioned [reference to neuralink], areas of the brain can be stimulated to improve perception, focus, receptivity, as to rise the mental capability of the individuals. Not only this, through electrodes and brain-waves scans, mental computation support and visualization interfaces could be implemented, also providing modules for further augmented reality (AR) and communication applications.\\

Immediate sharing of complex mental schemes, AR visualization of contextual information, and mental modelling of virtual reality… these possibilities would offer great empowerment to creativity based roles and jobs like police detectives, engineers and architects, mathematicians, philosophers, artists. Such enhancement would improve mutual human communication and understanding, for the benefit of every user’s human relationship.\\

Lastly, short-distance mobility could be positively changed, wise of augmentations affecting limbs functionalities and strength, for example. Also, integrating generic entertainment and utility components into the cyber parts would be possible.


\section*{Society}
\label{sec:society}

By a surface analysis of a plausible social picture where cyborg augmentations, as we devise them, are widespread, no particular disruptive elements are to be found. The looks of cyber individuals would very closely resemble the human ones, preventing any generation of visual discrimination. Rather, an individual may affordably customize its looks in the way that better fits his personality and desire, with very positive effects on self-confidence [cite studies on looks and self confidence?]. As said, social interactions also would benefit from augmentations, in the way in which brain interfaces shall provide an additional communication level, and improve sensitivity towards a social context - possibly analysing significant psychological elements. Those interfaces in fact, and an encouragement for a higher mindfulness, would represent a further instrument not to arouse incomprehensions.\\

Productivity, quality of life and digital adaptation are the main objectives. Remaining within the frame of the already existing western societal structure [Marx], further digital transformation would come but in the process of integrating augmentations. It would thus be easily manageable in a first place, while raising the living standards, possibly paving the way for further societal development, towards an ideological shift of paradigm for a less work-intensive, sustainable and mindful society.\\

Of course though, a global access divide between those who can and those who cannot afford or access to the augmentation technologies would tend to emerge. Making some individuals more apt to enjoy the new possibilities and their positive effects. Addressing this concern, in our vision, we wish and foresee international entities to regulate the market of augmentations, promoting open innovation, open source projects, research and initiatives to extend the enhancement services, especially the healthcare sector, as widely as possible [As in the case of the internet: cite examples of organizations!].

\section*{Overall Concerns}
\label{sec:concerns}

While many are the possibilities for a globally positive impact of cyber augmentations, some downsides and concerns have to be taken into account.\\

A very first note is that the installation of such components would have to be as little invasive as possible, so that a proper momentum for a similar market would be in place. And in case of components faults, the user must be very promptly provided with proper substitution in order to grant its bodily integrity. With respect to the offered functionalities, there have to be regulations to limit their scope of action, not to generate augmentations that would give an individual powers or offensive potential too complex to manage. Along the same consideration, bodily modifications would have to abide by ethical standards. A proper method to deal with rogue augmentation producers and augmented individuals would have to be set up.\\

Paramount attention has to be put in making sure that no augmentation can be exploited by the manufacturers or malicious parties. Especially for what concerns the brain interfaces and enhancements, hence the promotion of open source technology, research and transparency of implementation.\\

Any component has to work exclusively offline, via inputs that are not but sensory or cerebral. The only exception is with some augmented reality application and brain communication: they must be implemented in a way for which only visual information may be hacked, and a reliable turn-off switch must be present in case of extreme need.

\section*{Final Remark}
\label{sec:remark}

Augmentations of the cyborg type bear a set of advantages over the genetic modification ones. Humanity is preserved, in the looks and in the genetic identity. Besides, any possible genetic modification can be replicated via electromagnetism and synthetic materials, in a way which is faster, more flexible, and more scalable. And even more can be done. Algorithmic and AI power provided to brain interfaces and augmented reality can boost brain efficiency, enhance communication, and “life UX”.\\

Integrating the machine with the man, no human-computer dichotomy is to persist: societary technological disruptions are averted - and Terminator scenarios avoided. Like the advent of the internet, widespread cyber components will bring the human being to reach a higher evolutionary stage, the first pseudo-evolution after the sapiens-sapiens. The humanity will be stronger, more powerful, and socially intelligent [maybe cite a source].\\

The technology may be achieved, the rest is on the policies.


      \chapter*{Genetical Modified}
\label{cha:gm}

\section*{Relationship}
\label{sec:gm_relationship}
In the future, humanity will progressively lose its essence due to the continuous evolution and improvement of machines, in the long term, this could lead to machine automation and robotization as the core of everything, placing humans in a marginalized position. For this reason, Genetic Modification is the most suitable solution to keep human beings in a vital place. One of the main points is undoubtedly related to the impact on human relationships, how are the family and, more in general, society structure going to be affected? Nowadays, most of the time, relationships don’t work because people don’t understand each others’ feelings and points of view. One of the reasons behind this issue is the fact that, with the development of more and more advanced technologies, human beings are progressively shifting to experience alternative realities (e.g. virtual reality technology) and at the same time, they are getting far from real relationships, leading them to become increasingly isolated. With genetic modification, human beings will be able to increase their ability to have more empathy and solidarity with each other, this will achieve a deeper connection among them. A good example of this kind of environment is “The Incredibles”, who, although they have superpowers, they manage to keep the family structure and identity as we know it.

\section*{Work}
\label{sec:work}
Genetic Modification affects also the way we have seen the concept of work so far, thanks to it, humans can be smart enough to design and develop robots and systems that will perform all the repetitive tasks. Society, therefore, will encourage people to focus on jobs where creativity and innovation are the key skills and on areas where the performance of robots is quite limited such as design, movies, engineering, etc. Safety at work will considerably improve since professionals from different areas will receive correspondent security measures. For instance, they can be immune to radiation, which can be useful in the medical sector or nuclear sector. 

\section*{Health Care}
\label{sec:health}
The most powerful aspect in which Genetic Modification can affect society is health care. Diseases will no longer be a problem since for each one there will be an available treatment. Not only common diseases we can think about but also remarkable issues affecting the current society. are couples trying to start a family, in which one of the parents has fertility problems, for many of them, nowadays there is not a solution. Also, the process that trans people have to undergo to become the person they feel is very long and complex. Through Genetic Modification this can be easily solved in such a way that these people will feel better about the final result and more confident. 

Probably one of the biggest dangers for society right now is the emergence of antibiotic resistant bacteria. Due to the overuse and misuse of antibiotics, some bacteria are immune to every existent treatment and scientists have not been able to find a solution. Thanks to Genetic Modifications humans will be able to become resistant again. 

In addition to diseases, Genetic Modification is also able to treat accidents in which different organs or body parts are lost or damaged. By bio-reconstruction, the damaged or lost limb can be healed and reinserted in the body. In the worst case, when the mentioned body part cannot be recovered, bio-artificial organs will be used. These bio-artificial organs are done with cells coming from the own patient, so there is no risk of rejection and the feeling will be as the actual one. Finishing with the medical area, therapygenetics will also be covered. This is a science that treats mental diseases and the impact of genetics on them. By exploiting this area and introducing Genetic Modifications, many mental diseases will be easily solved. For instance, genetic illnesses like Alzheimer won’t be a problem anymore because the problem will be solved at the root. As a matter of fact, thanks to genetic modifications, for all the diseases will not only be a cure but will seem like that illness was never taken by the patient. So, putting all together, we will live not only in a healthier society but also happier.

\section*{Ethical boundaries}
\label{sec:ethics}
An essential issue to address when talking about Genetic Modification is ethical boundaries. Human society has always struggled in defining red lines that could find a compromise guaranteeing a peaceful coexistence between everyone. For this reason, Genetic Modification will be subject to regulations. This means that every requested modification - not related to health problems - must be justified, and the reason for the request has to meet requirements depending on different parameters. For example, if someone wants to change his eye color, he will find it easy. However, if he tries to become four meters tall this won’t be achievable since it affects the coexistence. The consequences of each modification have to be clearly pointed out and the individual will have to prove that he is informed about them. Moreover, everyone that wants to perform any change has to be assessed through a psychological test proving that he is in perfect psychological conditions and that he is not being forced by a third person to get that modification. Furthermore, even if someone decides to get a strange modification for his body, that does not mean that that kind of modification will be strange forever. Let’s just think about the social impact of tattoos at the beginning, it was strange for the majority of the population, but nowadays it is a normal thing. Finally, individuals under 18 years old won’t be allowed to get any Genetic Modification except for health purposes. 

\section*{Civil order}
\label{sec:order}
How to maintain civil order in such an unusual scenario? With respect to Genetic Modifications, some measures will be applied, but it has to be taken into account that without cooperation from both sides, they will not be useful. The measures consist of a strong rehabilitation process, where criminals undergo a modification helping them to increase their empathy and therefore become more conscious about society. This process also includes the learning of different skills that help them to reintegrate into society. By doing all this, we are choosing a system focused on reintegration rather than punishment, which in the long-term will tend to reduce criminality. However, since no plan can have 100\% satisfactory results, people with criminal records will have no longer access to Genetic Modifications that can be considered dangerous for society. In a similar way, those in charge of guarantee security will also take advantages of the new advancements. In order to live up to all the different kinds of modifications, the security corps have to go through hard training periods. Those ones, will not only give all the acknowledgment on possible Genetic Modifications of the population but will also coach them on how to handle their powers to make bad guys harmless. So, thanks to Genetic Modification the security corps will be a very well prepared team with powerful Genetic Modifications and the knowledge needed to use them accurately. 

\section*{Population}
\label{sec:population}
The last issue covering is population, which depends on both Genetic Modified humans and Cyborgs. It is not possible to determine the number of people getting each modification, and only some assumptions could be made. First of all, Genetic Modified humans have a life expectancy of around two hundred years. Although the possibility of immortality was an option, it was decided not to exploit it since the disadvantages overcome advantages - the main one is that human life would no longer seem like a normal life, losing its main purpose. Increasing life expectancy could end up with overpopulation. However, we have to take into account that these modifications are only available in developed countries where the real problem is the low birth rate, which leads to a lack of people. But, even if we suppose that at a certain point the problem of overpopulation arises, in the future where these technologies become more accessible, a possible solution is that everyone adopts Genetic Modifications to change the way digestion process takes place in order to obtain more energy and nutrients from less food. This will reduce planet exploitation. 

\section*{Remarks}
\label{sec:gm_remarks}
To sum up, in a world where all the technological improvements lead us to the new digital era of the human race, we can focus ourselves on body enhancements that cannot be reached by the robotization. As a matter of fact, almost every digital task can be reached without the need for implementing machines in our body; we can simply use all the machinery as devices we interact with. By doing so, we are allowed to reach all the benefits described above which will never be reachable by the improved digitalization of the human race. Finally, Genetic Modifications give us the opportunity to improve the creativity, the innovation, and all the other fields that cannot be reached by a simple predefined command executed very fast.

      
\chapter*{Reconciliation}
\label{cha:reconciliation}

\section*{Overview}
\label{sec:overview}

In a world where both cyborgs and genetically modified humans are living in the same place, forcing people to choose to go for one side or the other instead of taking the good of each one of them would certainly create a situation of inequality, which could lead to conflicts and tensions.\\

The two sides of the battle do not defend positions that are mutually exclusive, or symmetrical extremes of the subject of augmentation by themselves. It’s therefore important to work together to create a new scenario where people can freely choose, according to the situation, what kind of technology is the most suitable. From this conjunction point, our aim is thus that of exploring high level regulations and boundaries to maximize the added value of human augmentations, with respect to both the enhancement methods.\\

Imagine a world where everyone can permanently cure a genetic disease with a simple DNA modification, preventing this from emerging in future generations or imagine to fix a disability like legs dysfunction by replacing them in a few hours with a set of robotic legs that are identical to the normal one. Imagine a world where the concept of work is revolutionized, where humans can interact with machines and computer systems in a much more efficient way as if they’re talking to another human or a stage where humans can access and discover all the environments that were previously considered hazardous and inaccessible. Imagine a world where people can finally better understand each other, strengthen their relationships and rediscover themselves and who’s around them.\\

All of this can be achieved if society works side by side in strengthening and combining genetic modification with cyborgization. 

\section*{Demographics}
\label{sec:demographics}

As far as the shape and figures of the population are concerned, no significant opposing arguments would arise. It is reasonable to foresee that through enhancement technologies, people will be able to live much longer: the combination of cyber implants and biological and genetic engineering will render healthcare far more effective in general, slowing overall aging, eradicating several diseases, drastically quickening the response time to health issues and shortening medical recovery times.\\

While the age of the population will rise, so will be their average health and productivity over time, allowing for a general translation to a retribution and retirement system on a more extended period. Such systems being already present, it would certainly avoid disruption and system re-design needs, at least in an initial phase – opening for further developments and integration with automation and wealth distribution possibilities.\\

Concerning the issue of managing a possible overpopulation, as the Genetic Modifications team proposes, optimizing the digestion process could grant the sustainability of global nutrition assets. Besides food, many other societal issues come with overpopulation – managing the job market, the living conditions and the psycho-social implications of highly populated areas are among those. Well devised birth control policies, redistribution and work opportunities policies, urban plans, as well as the collective smartness “engine” that cyber communication may instantiate, could tackle the issue in a way.\\

Indeed, what discussed here assumes a uniform adoption of enhancement technologies. This would be the case if (at least) healthcare-related augmentations are universally available: again, a matter of policy definition and actuation that we have cognition of, so that no evident class A and class B human division is in place.

\section*{Market implications}
\label{sec:market}

Enhancement technology was first born for medical applications [source] and later adapted to further uses. It seems fair to assume that the augmentation market would thus be divided in two main categories: healthcare sector (enhancement to normalize or prevent a deficit) and all the remaining (cognitive and physical enhancements for healthy individuals) [source]. Within the battle constraints of hard-coded mutual exclusion among the two augmentation types, in order to gain appreciation and competitiveness, each side would have had to provide capabilities to reach for both markets. Pragmatically speaking, there is no reason to assume so.\\

Cyber micro-cells could aid in performing biological modification routines, whereas specialized bio-engineering may make the neural communication between body and cyber components more efficient. While cyber parts wouldn’t get illnesses, very much could the rest of the non-cyborg body, so genetic modification-based medications would be of great need in any case. Under a different perspective, having a lost limb or a failing organ regrown would bear disadvantages over having it substituted right away with a synthetic, stronger one.\\

On the second market domain, applications in micro-mobility better suits the cyborg case; overall bodily strength, environmental resilience and adaptation possibilities would scale smoothly with cyber augmentations, but bones and skin reinforcing would provide for a less invasive and general purpose first level of integrity enhancement. Cognitive empowerment would also work on different levels: genetic drugs would provide a less invasive approach to improve focus, wit and perception, while electrodes-based digital brain interfaces would open huge human-human and human-machine communication possibilities, as well as cognitive and computational capabilities to apply in working contexts.\\

With the discussed market subdivision, defining a class of medical-use augmentations would be the first step towards an integration of those in national and international healthcare services; while other context-specific regulations can be devised in the other market.


\section*{Overall Societal Concerns}
\label{sec:rec_concerns}

There is no denial that augmentation technologies will tend to generate a productivity and life quality divide among those who possess them, and those who do not. This is true especially considering the value that an augmented individual has for in the job market over the non augmented people. Once again both sides in the battle suffer from this condition.\\

A way to address such issue is by considering one among three regulatory approaches for augmentation in the work sector: top-down, where it is the Government or a company that provides its employees with augmentations; bottom up, where it is up to each individual; or a mixed approach of the two [cite augmentations and work study]. Both teams reckon that a top-down approach would be sensible where augmentations are not permanent: with the assumption that genetic modification technologies are 100\% reliable, temporary brain and physical enhancements would very well serve the case where an organization offers his workers such option, in order to gain advantage over competitors, and not generating divide among the human workforce.\\

Several are, though, the possibilities of enhanced augmented reality and connectivity that the cyber augmentations offer. Deeper communication levels and human-machine interaction will possibly be of huge value in sectors like engineering and management. Because of ethical issues related to body integrity, a top-down approach would be harder to implement (unless such technologies are detached from the body), but a bottom-up would generate too much divide. Autonomous workers aside, an option for the mixed approach would be a policy for coupling the positions with augmented and non augmented individuals. In order to further empower workers’ rights in a so delicate position, the just mentioned policy can be accompanied by a reform of the trade union, so that the introduction of augmentation shall also address a better protection of workers from any form of exploitation. Addressing the general issue of access divide, we refer to the general considerations each side brought in its view.\\

Further concerns include the constitution of a black and illicit augmentation market. The subject is still shared between the sides: special forces are needed to tackle possible misuse of the technologies. As it happens nowadays, there is no a magic formula to fight against black market; however, strong policies and a specialized security corp will be key to combat it and lead it to its lowest levels. \\

One last point of debate is how to maintain a factual identity metric when body parts are substituted and genes altered. Trace of the mutated genetic information related to an identifying label of an individual (its name) may be kept from the birth (maybe through the use of a blockchain) and stored in a national database. Such data object information can then be physically integrated into the cyber components that an individual may acquire.


\section*{Human relationships}
\label{sec:relationships}

One of the most critical aspects that human enhancement - and all that gravitates around it - touches is certainly the sphere of human relationships. \\

In this field it is important more than ever that cyborg and genetic modification technologies combine their features in the best possible way. This aspect needs to be stressed not only to avoid the creation of unhealthy relationships or inequality, but also to sharply improve the current situation of human relationships.

Nowadays, in this strongly work-money-based society, human beings struggle very often in understanding each other. This happens not only because of the ordinary issues that occur, for instance, within family or couples, but because mental illness - caused by work-related stress and anxiety -  is increasingly becoming a big and central issue. In our future context, improving individual’s quality of life will be a primary aim of augmentations. The endeavour through which such technologies are conveyed shall push towards empathy and fairness. Cognitive enhancement will especially address mental illnesses, and forms of stress that may derive from work and similar environments. This would be achieved in the practical aspect of enhancing brain efficiency and cognitive effort management, and, consequently, with the promise of granting more free time to enjoy and pursue whatever life interest an individual may have.\\

The above, together with inclusion policies, will tackle the issue of psychological subjection and envy that may possibly arise in non augmented individuals, with the hope of smoothing out communication and relationship gaps between anyone, in such shaped society.\\

Cosmetics as well will play a role in relationships: addressing the issue of psychological ease with respect to visual perception and identification of peers (ethnic identity in a future multicultural context) augmentations would have to be aesthetically accepted. The teams agree that physically enhancing technologies (those overall targeting strength and mobility) best apply to the cyborg case, as the provided augmentations could be customized in looks, scalable and easily substituted. On the other hand, “beauty” augmentations – the possibility to change one’s facial and body traits to meet their desires – are best targeted by highly engineered genetic modification means.

      
\chapter*{Conclusion}
\label{cha:conclusion}

As the State of the digital and medical Arts progresses on, and the state of the globalized society moves towards an increasingly deeper self-consciousness [there is a source saying this], we may foresee the topic of human augmentations gaining a progressively larger focus. Once again, our sides in the battle - genetic modification and cyborg augmentations - do not constitute mutually exclusive: our reconciliation expresses an effort to highlight some of the most relevant concerns that devising an augmented society should consider.\\

Points of discussion have been provided on the innovative potential that both augmentation technologies bear. Emerging in an incremental fashion from the State of Art basis, enhancing the cognitive and physical capabilities, providing advanced computational and visualization power to the individuals, they offer disruptive possibilities with respect to the actual automation and digitalization trends.  But again, the disruption related to the mere “factual” introduction of such technologies is not set to be so drastic, given how acquainted the society got with innovation: an initial post-digital transformation that human augmentations entail could be smooth. What instead smooth wouldn’t be, as has been noted multiple times throughout the report, is the implementation of a policy structure that should come with enhancements.\\

One point of debate dominates the concerns on augmentations: the divide, the wealth and opportunities gap they may generate [cite the reports on augmentation and work, economic inequality, etc.]. With this knowledge, we reckon that the focus on policies has paramount importance in the matter, possibly more than the augmentation possibilities themselves - and it is, in fact, rather difficult to fully explore. A business-as-usual approach may bear high risks of privileging the richer in enjoying augmentations, bringing about a massive societal division, based on wealth and power, between super-humans and sub-humanized individuals [cite economic inequality and HET]. Fundamental is thus that governments and international entities provide a deep regulatory frame, managing augmentation provisions in working sectors, in healthcare, and augmentation scope of functionalities. Corporate Social Responsibility would be mandatory for any venture participating in the enhancement market, and such products will have to be user and society-oriented. As has been the case for the internet, augmentations would represent a new stage for humanity, and we better get it right.

   
      
    \endgroup



    \bibliographystyle{plain}
    \bibliography{biblio}

\end{document}
